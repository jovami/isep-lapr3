% vim:foldmethod=marker spelllang=pt,en
\documentclass[12pt, a4paper]{article}

% Preamble {{{

% Packages {{{

% lang & encoding
\usepackage[utf8]{inputenc}
\usepackage[portuguese]{babel}
\usepackage{csquotes}
\usepackage[T1]{fontenc}
\usepackage{hyphenat}
\usepackage{textcomp}

% images
\usepackage{graphicx}
\graphicspath{ {./img/} }

% index
\usepackage{imakeidx}
\usepackage[inline]{enumitem}

% layout and navi
\usepackage{indentfirst} % indent first line of a section
\usepackage[
    hidelinks,
    colorlinks=true,
    linkcolor=blue
]{hyperref}
\usepackage[export]{adjustbox}
\usepackage[
    left=2cm,
    right=2cm,
    top=2cm,
    bottom=2cm
]{geometry}

% }}}

\renewcommand{\baselinestretch}{1.3}
\makeindex

% Keywords
\providecommand{\keywords}[1]
{
    \small
    \textbf{Keywords:} #1
}

% bib
% \usepackage[backend=biber, style=ieee,urldate=iso,seconds=true,dateabbrev=false]{biblatex}
% \addbibresource{refs.bib}

% fileinfo
\title{Dicion\'ario do Modelo de Dados (2DD, Groupo 31)}

\author{
    Marco Maia          {\textemdash}    1210951\\
    R\'uben Ferreira    {\textemdash}    1210954\\
    Jo\~{a}o Teixeira   {\textemdash}    1210957\\
    Jos\'e Rente        {\textemdash}    1211155\\
}

% }}}

% start
\begin{document}

% title page
\makeatletter
\begin{titlepage}
    \begin{center}
        \par
        \noindent
        \includegraphics[
            width=0.3\textwidth,
            valign=M,
            margin=0ex 2ex
        ]{isep-logo.png}
        \hfill
        \includegraphics[
            width=0.3\textwidth,
            valign=M,
            margin=0ex 2ex
        ]{dei-logo.png}\\[4ex]
        \par
        {\bfseries  \@title}\\[2ex]
        {\@date}\\[2ex]
        {\@author}
    \end{center}
\end{titlepage}
\makeatother
\thispagestyle{empty}
\newpage

{
    \hypersetup{hidelinks}
    \tableofcontents
    \printindex
    \newpage
}

\section{Introdu\c{c}\~ao}\label{sc:intro}

O presente documento consiste num dicion\'ario de dados, criado com o intuito de detalhar os aspetos mais relevantes dos elementos da base de dados.

%% mais algo aq?

% TODO: sort entries alphabetically and fix spelling mistakes

\section{Dicion\'ario}\label{sc:dict}

%%

\subsection{Caderno de Campo}\label{ssc:cad_campo}

Entidade que modela o conceito de \emph{Caderno de Campo}, introduzido pelo Cliente.

Um caderno de campo trata-se de um documento formal que permite registar todas as
opera\c{c}\~oes agr\'{\i}colas relevantes que ocorreram na explora\c{c}\~ao.

%%

\subsection{C\'odigo Postal}\label{ssc:codigo_post}

O c\'odigo postal trata-se de uma sequ\^encia alfanum\'erica utilizada
para permitir a identifica\c{c}\~ao de endereços.

% \begin{itemize}
%     \item \textbf{Primary Key:}
%         codigo{\_}postal{\_}id
% \end{itemize}

%%

\subsection{Composto}\label{ssc:composto}

Entidade que modela um componente que pertence \`a da constitui\c{c}\~ao de um
fator de produ\c{c}\~ao~(\ref{ssc:fator_prod}).

%%

\subsection{Dado Meteorol\'ogico}\label{ssc:dado_met}

Para melhorar o processo de gest\~ao de culturas \'e fulcral registar a informa\c{c}\~ao
meteorol\'ogica, como
\begin{enumerate*}[label = (\roman*)]
    \item a \textbf{temperatura},
    \item a \textbf{humidade do ar},
    \item a \textbf{pluviosidade}
        e
    \item a \textbf{velocidade do vento}
\end{enumerate*}.

%%

% NOTE: table removed
% \subsection{Dep\'osito}\label{ssc:deposito}

% Um condutor deposita entregas em \textbf{Hubs} (\ref{ssc:hub}),
% sendo o Gestor de Distribuição quem as respetivas entregas.

% \begin{itemize}
%     \item \textbf{Primary Key:} Chave concatenada dos seguintes atributos:
%         \begin{itemize}
%             \item
%                 data{\_}deposito,
%             \item
%                 condutor{\_}id,
%             \item
%                 hub{\_}id,
%             \item
%                 gestor{\_}distribuição
%         \end{itemize}
%     \item \textbf{Foreign Keys:}
%         \begin{itemize}
%             \item
%                 condutor{\_}id,
%             \item
%                 hub{\_}id,
%             \item
%                 gestor{\_}distribuição
%         \end{itemize}
% \end{itemize}

%%

\subsection{Edif\'{\i}cio}\label{ssc:edificio}

Locais onde s\~ao armazenados os diversos componentes fulcrais para a produ\c{c}\~ao
agr\'{\i}cola como, por exemplo,
\begin{enumerate*}[label = (\roman*)]
    \item \textbf{tratores}
    \item \textbf{fertilizantes}~(\ref{ssc:fert})
        ou, at\'e mesmo,
    \item \textbf{produtos resultantes da explora\c{c}\~ao agr\'{\i}cola}~(\ref{ssc:prod})
\end{enumerate*}.

%%

\subsection{Encomenda}\label{ssc:encomenda}

Quando um cliente pretende comprar um ou v\'arios produtos~(\ref{ssc:prod}),
uma encomenda \'e criada.

Esta ter\'a um \textbf{valor}, uma \textbf{data limite de pagamento},
bem como um \textbf{endere\c{c}o de entrega}.

% \begin{itemize}
%     \item \textbf{Primary Key:} Chave concatenada dos seguintes atributos:
%         \begin{itemize}
%             \item
%                 data{\_}estimada{\_}entrega,
%             \item
%                 gestor{\_}agricola{\_}id,
%             \item
%                 cliente{\_}id
%         \end{itemize}
%     \item \textbf{Foreign Key:}
%         hub{\_}id
% \end{itemize}

%%

\subsection{Encomenda{\textendash}Produto}\label{ssc:encomenda_prod}

Cada \textbf{Encomenda}~(\ref{ssc:encomenda}) pode conter um ou
m\'ultiplos \textbf{Produtos}~(\ref{ssc:prod}).

Nesta entidade, ser\'a armazenada a informação relativa \`a
lista de produtos de cada encomenda.

% \begin{itemize}
%     \item \textbf{Primary Key:} Chave concatenada dos seguintes atributos:
%         \begin{itemize}
%             \item
%                 gestor{\_}agricola{\_}id,
%             \item
%                 cliente{\_}id
%             \item
%                 data{\_}estimada{\_}entrega
%             \item
%                 produto{\_}id
%         \end{itemize}
%     \item \textbf{Foreign Keys:}
%         \begin{itemize}
%             \item
%                 gestor{\_}agricola{\_}id,
%             \item
%                 cliente{\_}id
%             \item
%                 data{\_}estimada{\_}entrega
%             \item
%                 produto{\_}id
%         \end{itemize}
% \end{itemize}

%%

\subsection{Estado da Encomenda}\label{ssc:encomenda_state}

Como foi mencionado na Subsec\c{c}\~ao~\ref{ssc:encomenda_reg}, a encomenda pode-se
encontrar em tr\^es estados diferentes:
\begin{enumerate*}[label = (\roman*)]
    \item \textbf{Registada},
    \item \textbf{Entregue}
        ou
    \item \textbf{Paga}
\end{enumerate*}.

% \begin{itemize}
%     \item \textbf{Primary Key:}
%         estado{\_}encomenda{\_}id
% \end{itemize}

%%

\subsection{Fator de Produ\c{c}\~ao}\label{ssc:fator_prod}

Um fator de produ\c{c}\~ao consiste todo o produto que possa ser aplicado no solo ou nas
plantas da instala\c{c}\~ao, de forma a melhorar a qualidade do solo, a nutrir as plantas,
prevenir doen\c{c}as, combater pragas, entre outros.

%%

\subsection{Fertiliza\c{c}\~ao}\label{ssc:fert}

Entidade que modela o processo de fertiliza\c{c}\~ao. Entende-se por fertiliza\c{c}\~ao
como a aplica\c{c}\~ao de um fator de produ\c{c}\~ao~(\ref{ssc:fator_prod}) numa
parcela agr\'{\i}cola~(\ref{ssc:parcela_agr}).

%%

\subsection{Ficha T\'ecnica}\label{ssc:ficha}

Ficha que cont\'em a as subst\^ancias contidas no
fator de produ\c{c}\~ao~(\ref{ssc:fator_prod}), bem como as respetivas quantidades.

%%

\subsection{Formula\c{c}\~ao}\label{ssc:formulacao}

Um fator de produ\c{c}\~ao~(\ref{ssc:fator_prod}) pode assumir diferentes estados,
\begin{enumerate*}[label = (\roman*)]
    \item \textbf{l\'{\i}quido},
    \item \textbf{granulado}
        ou
    \item \textbf{em p\'o}
\end{enumerate*}, dependendo do \textbf{tipo de fator}~(\ref{ssc:tipo_fator_prod})
em concreto e/ou do fabricante.

%%

\subsection{Gestor Agr\'{\i}cola}\label{ssc:gest_agr}

Um gestor agr\'{\i}cola desempenha a função de gerir as culturas nas
parcelas~(\ref{ssc:parcela_agr}), realizar as diferentes ações culturais e regist\'a-las no
\textbf{Caderno de Campo}~(\ref{ssc:cad_campo}), bem como gerir os
pedidos~(\ref{ssc:encomenda}) efetuados pelos Clientes.

% \begin{itemize}
%     \item \textbf{Primary Key:} Chave concatenada dos seguintes atributos:
%         \begin{itemize}
%             \item
%                 gestor{\_}agricola{\_}id,
%             \item
%                 instalacao{\_}agricola{\_}id,
%             \item
%                 data{\_}inicio{\_}contrato,
%             \item
%                 data{\_}fim{\_}contrato
%         \end{itemize}
%     \item \textbf{Foreign Keys:}
%         \begin{itemize}
%             \item
%                 gestor{\_}agricola{\_}id,
%             \item
%                 instalacao{\_}agricola{\_}id
%         \end{itemize}
% \end{itemize}

%%

\subsection{Hub}\label{ssc:hub}

Local onde as encomendas~(\ref{ssc:encomenda}) entregadas s\~ao depositadas e,
posteriormente, recolhidas pelos clientes.

% \begin{itemize}
%     \item \textbf{Primary Key:}
%         hub{\_}id
%     \item \textbf{Foreign Key:}
%         codigo{\_}postal{\_}id
% \end{itemize}

%%

\subsection{Incidente}\label{ssc:incidente}

Entidade que modela o evento de quando, por algum motivo,
um cliente que fez uma encomenda~(\ref{ssc:encomenda})
n\~ao p\^ode efetuar o pagamento até uma determinada data limite.

% \begin{itemize}
%     \item \textbf{Primary Key:}
%         incidente{\_}id
%     \item \textbf{Foreign Key:}
%         cliente{\_}id
% \end{itemize}

%%

\subsection{Instalacao Agr\'{\i}cola}\label{ssc:inst_agr}

Regi\~ao composta por v\'arios edif\'{\i}cios~(\ref{ssc:edificio}) e setores~(\ref{ssc:parcela_agr}).

%%

\subsection{Instalacao Agr\'{\i}cola{\textendash}Fator de Produ\c{c}\~ao}\label{ssc:inst_agr_fat_prod}

Como cada instala\c{c}\~ao agr\'{\i}cola~(\ref{ssc:inst_agr}) utiliza uma s\'erie de
fatores de produ\c{c}\~ao~(\ref{ssc:fator_prod}) e em diferentes quantidades, \'e
bastante importante registar esta movimenta\c{c}\~ao dentro de cada instala\c{c}\~ao
agr\'{\i}cola, de modo a melhor ger\'{\i}-la.

%%

\subsection{Pagamento}\label{ssc:pagamento}

Um cliente pode pagar a sua \textbf{Encomenda} (\ref{ssc:encomenda})
por inteiro ou parcelada.

Para tal, aquando do pagamento, este escolhe o valor que pretende pagar,
sendo registada a data em que este foi efetuado.

% \begin{itemize}
%     \item \textbf{Primary Key:}
%         pagamento{\_}id
%     \item \textbf{Foreign Keys:}
%         \begin{itemize}
%             \item
%                 data{\_}estimada{\_}entrega,
%             \item
%                 cliente{\_}id,
%             \item
%                 gestor{\_}agricola{\_}id
%         \end{itemize}
% \end{itemize}

%%

\subsection{Parcela Agr\'{\i}cola}\label{ssc:parcela_agr}

Entidade modela o local onde decorre a planta\c{c}\~ao das diversas culturas;
i\.e\., dos diversos produtos~(\ref{ssc:prod}).

%%

\subsection{Plano de Rega}\label{ssc:plano_rega}

Para manter com regularidade a boa qualidade dos produtos~(\ref{ssc:prod}) produzidos e,
em simult\^aneo, reduzir os custos de produ\c{c}\~ao, \'e de grande import\^ancia manter
um plano de rega e de registar a hora quando as opera\c{c}\~oes de rega devem ocorrer.

%%

\subsection{Produto}\label{ssc:prod}

Entidade que modela aquilo que foi produzido nas culturas
de cada parcela agr\'{\i}cola~(\ref{ssc:parcela_agr}) e que posteriormente
ser\'a vendido a um cliente.

%%

\subsection{Rega Executada}\label{ssc:rega_exec}

Como nem todas as regas inclu\'{\i}das no plano de rega~(\ref{ssc:plano_rega}) poder\~ao ser
executadas~{\textemdash} por exemplo, se chover n\~ao ser\'a necess\'ario acionar
o sistema de rega~{\textemdash}, \'e necess\'ario manter um registo fi\'avel de todas
as regas que realmente decorreram.

%%

\subsection{Registo de Cultura}\label{ssc:reg_cult}

Entidade que modela o registo de todos os processos de planta\c{c}\~ao e colheita, bem como
as \textbf{data} em que ocorreram e a \textbf{quantidade produzida}.

%%

\subsection{Registo de Encomenda}\label{ssc:encomenda_reg}

Uma \textbf{Encomenda} (\ref{ssc:encomenda}) pode-se encontrar
em diferentes \textbf{Estados} (\ref{ssc:encomenda_state}),
sendo, portanto, crucial guardar a informa\c{c}\~ao relativa
a cada um destas, bem como a data em que ocorreu a transi\c{c}\~ao
entre cada um dos estados.

% \begin{itemize}
%     \item \textbf{Primary Key:} Chave concatenada dos seguintes atributos:
%         \begin{itemize}
%             \item
%                 estado{\_}encomenda{\_}id,
%             \item
%                 gestor{\_}cliente{\_}id,
%             \item
%                 cliente{\_}id
%             \item
%                 data{\_}estimada{\_}entrega
%         \end{itemize}
%     \item \textbf{Foreign Keys:}
%         \begin{itemize}
%             \item
%                 estado{\_}encomenda{\_}id,
%             \item
%                 gestor{\_}cliente{\_}id,
%             \item
%                 cliente{\_}id
%             \item
%                 data{\_}estimada{\_}entrega
%         \end{itemize}
% \end{itemize}

%%

\subsection{Tipo de Composto}\label{ssc:tipo_composto}

Cada composto~(\ref{ssc:composto}) pode ser de um de dois tipos:
\begin{enumerate*}[label = (\roman*)]
    \item \textbf{elementos}
        ou
    \item \textbf{subst\^ancias}
\end{enumerate*}.

%%

\subsection{Tipo de Cliente}\label{ssc:tipo_cliente}

Cada cliente pode pertencer a um de tr\^es n\'{\i}veis (tipos) {\textemdash}
A, B ou C {\textemdash},
consoante o seu \textbf{n\'umero de incidentes} (\ref{ssc:incidente})
e \textbf{volume total de encomendas pagas} (\ref{ssc:encomenda_reg})

% \begin{itemize}
%     \item \textbf{Primary Key:} tipo{\_}cliente{\_}id
% \end{itemize}

%%

\subsection{Tipo de Cultura}\label{ssc:tipo_cult}

Entidade que descreve a perman\^encia de uma cultura; i\.e\., se a cultura de um determinado
produto~(\ref{ssc:prod}) de uma parcela~(\ref{ssc:parcela_agr}) \'e permanente ou tempor\'aria.

%%

\subsection{Tipo de Edif\'{\i}cio}\label{ssc:tipoedificio}

Cada edif\'{\i}cio possui um tipo, j\'a que cada um tem uma determinada fun\c{c}\~ao
dentro de uma instala\c{c}\~ao, tais como
\begin{enumerate*}[label = (\roman*)]
    \item \textbf{garagem}
    \item \textbf{est\'abulo}
\end{enumerate*}, entre outros.

%%

\subsection{Tipo de Fator de Produ\c{c}\~ao}\label{ssc:tipo_fator_prod}

Cada fator de produ\c{c}\~ao tem um certo tipo como, por exemplo,
\begin{enumerate*}[label = (\roman*)]
    \item \textbf{adubo org\^anico},
    \item \textbf{fertilizante}
\end{enumerate*}, entre outros.

%%

\subsection{Tipo de Fertiliza\c{c}\~ao}\label{ssc:tipo_fert}

Cada fertiliza\c{c}\~ao pode decorrer de diferentes maneiras, tais como
\begin{enumerate*}[label = (\roman*)]
    \item a \textbf{fertirriga\c{c}\~ao}
        e
    \item a \textbf{rega foliar}
\end{enumerate*}

%%

\subsection{Tipo de Rega}\label{ssc:tipo_rega}

Entidade que modela os diversos tipos de rega~(\ref{ssc:rega_exec}), tais como
\begin{enumerate*}[label = (\roman*)]
    \item \textbf{gotejamento}
        e
    \item a \textbf{aspre\c{c}\~ao}
\end{enumerate*}.

%%

\subsection{Tipo de Sistema}\label{ssc:tipo_sys}

Entidade que modela os diversos sistemas de rega~(\ref{ssc:rega_exec}), tais como
\begin{enumerate*}[label = (\roman*)]
    \item a \textbf{rega bombeada}
        e
    \item a \textbf{rega por gravidade}
\end{enumerate*}.

%%

\subsection{Tipo de Utilizador}\label{ssc:tipo_user}

Cada utilizador tem uma função associada, sendo que esta pode ser uma de quatro tipos:

\begin{enumerate}
    \item \textbf{Cliente}
    \item \textbf{Gestor Agr\i{\i}cola}
    \item \textbf{Condutor}
    \item \textbf{Gestor Distribui\c{c}\~ao}
\end{enumerate}

% \begin{itemize}
%     \item \textbf{Primary Key:} tipo{\_}utilizador{\_}id
% \end{itemize}

%%

\subsection{Sensor}\label{ssc:sensor}

Entidade que modela o objeto com a fun\c{c}\~ao de gerar os
dados meteorol\'ogicos~(\ref{ssc:dado_met}).

%%

\subsection{Stock}\label{ssc:stock}

Entidade que modela o registo das movimenta\c{c}\~oes dos produtos~(\ref{ssc:prod}) produzidos.

%%

\subsection{Utilizador}\label{ssc:user}

Todos os utilizadores do sistema t\^em um tipo de utilizador associado,
que indica a fun\c{c}\~ao do mesmo no contexto do projeto desenvolvido.

Cada utilizador tem um s\'o tipo de função~(\ref{ssc:tipo_user}), não sendo assim poss\'{\i}vel
ter um utilizador a desempenhar m\'ultiplas fun\c{c}\~oes.

% \begin{itemize}
%     \item \textbf{Primary Key:}
%         utilizador{\_}id {\textemdash} identificador gerado de forma autom\'atica
%     \item \textbf{Foreign Keys:}
%         tipo{\_}utilizador{\_}id,
%         tipo{\_}cliente{\_}id,
%         codigo{\_}postal{\_}id
% \end{itemize}

%%

\end{document}
