\documentclass[12pt, a4paper]{article}

% Preamble {{{

% Packages {{{

% lang & encoding
\usepackage[utf8]{inputenc}
\usepackage[portuguese]{babel}
\usepackage{csquotes}
\usepackage[T1]{fontenc}
\usepackage{hyphenat}
\usepackage{textcomp} % ordinals: º && ª

% images
\usepackage{graphicx}
\graphicspath{ {./img/} }

% index
\usepackage{imakeidx}
\usepackage[inline]{enumitem}

% tables
\usepackage{tablefootnote}
\usepackage{multirow}
\usepackage{booktabs}

% symbols
\usepackage{gensymb}
\usepackage{amsmath}

% layout and navi
\usepackage{indentfirst} % indent first line of a section
\usepackage[hidelinks]{hyperref}
\usepackage[export]{adjustbox}
\usepackage[
    left=2cm,
    right=2cm,
    top=2cm,
    bottom=2cm
]{geometry}

% }}}

\renewcommand{\baselinestretch}{1.3}
\makeindex

% Keywords
\providecommand{\keywords}[1]
{
    \small
    \textbf{Keywords:} #1
}

% bib
\usepackage[backend=biber, style=ieee,urldate=iso,seconds=true,dateabbrev=false]{biblatex}
\addbibresource{refs.bib}

% fileinfo
\title{
    Resist\^{e}ncia e Energia {\textemdash} T\'ermica
}

\author{
    Marco Maia          {\textemdash}    1210951\\
    R\'{u}ben Ferreira  {\textemdash}    1210954\\
    Jo\~{a}o Teixeira   {\textemdash}    1210957\\
    Jos\'{e} Rente      {\textemdash}    1211155\\
}

% }}}

% start
\begin{document}

% title page
\makeatletter
\begin{titlepage}
	\begin{center}
		\par
		\noindent
		\includegraphics[
			width=0.3\textwidth,
			valign=M,
			margin=0ex 2ex
		]{isep-logo.png}
		\hfill
		\includegraphics[
			width=0.3\textwidth,
			valign=M,
			margin=0ex 2ex
		]{dei-logo.png}\\[4ex]
		\par
		{\bfseries  \@title }\\[2ex]
		{\@date}\\[2ex]
		{\@author}
	\end{center}
\end{titlepage}
\makeatother
\thispagestyle{empty}
\newpage

\tableofcontents
\printindex
\newpage

\section{Introdução}\label{sec:intro}

% section Introdução (end)

\section{Seleção de materiais}\label{sec:select}

Perante o problema apresentado, investigar um conjunto de materiais para fazerem parte de uma
estrutura, iniciou-se uma pesquisa em busca das melhores alternativas. Para tal, procurou-se
materiais com um \textbf{baixo valor de condutividade térmica ($k$)}.

\subsection{Paredes Exteriores}\label{sec:pext}

Perante uma situação de diferentes temperaturas nas diversas secções da estrutura, optou-se por manter a consistência
e utilizar os \textbf{mesmos materiais em todas as paredes exteriores}.

No final, obteve-se uma espessura de $32cm$.

\subsubsection{Camada Exterior}\label{sec:pext_ce}

Para a camada exterior das paredes, escolheu-se o \textbf{cimento}. Este material é usado em infraestruturas de todo o mundo
dado, não só às suas \textbf{características térmicas satisfatórias} mas, também, ao seu \textbf{baixo custo}.

Optou-se pela seguinte disposição:

\begin{center}
	\begin{tabular}{||c c c||}
		\hline
		Material & $k \hspace{1mm} (Wm^{-1}K^{-1})$ & $\Delta x \hspace{1mm} (m)$ \\ [0.5ex]
		\hline\hline
		Cimento  & $0,46$                           & $0,09$                      \\
		\hline
	\end{tabular}
\end{center}

\subsubsection{Camada Isolante e Estrutural}\label{sec:pext_cie}

Para a camada isolante e estrutural, destacaram-se os seguintes materiais:

\begin{center}
	\begin{itemize}
		\item ICF;
		\item Tijolo refratário, ($k=0,78 \hspace{1mm} Wm^{-1}K^{-1}$).
	\end{itemize}
\end{center}

\begin{figure}[htpb]
	\centering
	\includegraphics[width=0.3\textwidth]{icf-exemplo.png}
	
	\caption{Sistema ICF, ainda por preencher com betão armado}\label{fig:icf1}
\end{figure}

Entre ambos, foi decidido utilizar o ICF.\@ O ICF é uma sistema de construção distinguido pelo seu elevado
\textbf{isolamento térmico} e acústico, baixo custo de manutenção e fácil aplicação. Este sistema é constituído por
\textbf{dois blocos isolantes verticais} de \textbf{poliestireno expandido} que, após a sua respetiva montagem,
são preenchidos por \textbf{betão armado}.


Tendo sido desenvolvidos há pouco mais de 30 anos, estes sistemas têm sido utilizados um pouco por todo o mundo,
com especial ênfase nos EUA e no Canadá, dadas as suas ótimas capacidades \textbf{térmicas} e acústicas

Apesar de, na figura~\ref{fig:icf1}, estar representado um reforço com barras de metal, essas serão
ignoradas neste trabalho experimental.

Optou-se pela seguinte disposição:

\begin{center}
	\begin{tabular}{||c c c||}
		\hline
		Material               & $k \hspace{1mm} (Wm^{-1}K^{-1})$ & $\Delta x \hspace{1mm} (m)$ \\ [0.5ex]
		\hline\hline
		Poliestireno Expandido & $0,037$                          & $0,02$                      \\
		\hline
		Betão Armado           & $2$                              & $0,18$                      \\
		\hline
		Poliestireno Expandido & $0,037$                          & $0,02$                      \\
		\hline
	\end{tabular}
\end{center}

\subsubsection{Camada Interior}\label{pext_ci}

Para a camada interior, destacaram-se os seguintes materiais:

\begin{center}
	\begin{itemize}
		\item Gesso, ($k=0,25 \hspace{1mm} Wm^{-1}K^{-1}$);
		\item Estuque, ($k=0,4 \hspace{1mm} Wm^{-1}K^{-1}$).
	\end{itemize}
\end{center}

Pelas claras diferenças nos valores de condutividade térmica, escolheu-se o \textbf{gesso} para o revestimento interior
das paredes exteriores.

Optou-se pela seguinte disposição:

\begin{center}
	\begin{tabular}{||c c c||}
		\hline
		Material & $k \hspace{1mm} (Wm^{-1}K^{-1})$ & $\Delta x \hspace{1mm} (m)$ \\ [0.5ex]
		\hline\hline
		Gesso    & $0,25$                           & $0,01$                      \\
		\hline
	\end{tabular}
\end{center}


\subsection{Paredes intermedia isolantes}\label{sub:Paredes intermedia isolantes}

% subsection Paredes intermedia isolantes (end)


\subsection{Paredes interiores}\label{sub:Paredes interiores}

% subsection Paredes interiores (end)

\subsection{Telhado}\label{sub:Telhado}

Para o telhado, optou-se por um modelo de duas águas. Para a \textbf{estrutura exterior}, o que fez mais sentido foi
uma cobertura de \textbf{cimento}, sobreposto por uma camada de \textbf{telha}. O cimento, pelas mesmas razões referidas no tópico
\ref{sec:pext_ce}, foi a melhor decisão, dado aos seus baixos valores de condutividade térmica de $k=0,46 \hspace{1mm} Wm^{-1}K^{-1}$.

Já o \textbf{material isolante} escolhido, difere do material isolante das paredes exteriores. Optou-se
por \textbf{espuma de poliuretano} que, permite obter um isolamento térmico que satisfaz as necessidades
do caso de estudo. Este material apresenta um valor de condutividade térmica de
$k=0,028 \hspace{1mm} Wm^{-1}K^{-1}$ e, é muito popular nas indústrias que dependem de \textbf{espaços
	com temperaturas controladas}.

\begin{figure}[htpb]
	\centering
	\includegraphics[width=0.3\textwidth]{espuma_poliuretano.jpg}
	\caption{Espuma de poliuretano}\label{fig:poliuretano}
\end{figure}

Por fim, e, à semelhança das paredes exteriores, decidiu-se aplicar uma camada de \textbf{gesso}, como
\textbf{revestimento interior do telhado}.

Optou-se pela seguinte disposição:

\begin{center}
	\begin{tabular}{||c c c c||}
		\hline
		Estrutura & Material              & $k \hspace{1mm} (Wm^{-1}K^{-1})$ & $\Delta x \hspace{1mm} (m)$ \\ [0.5ex]
		\hline\hline
		Exterior  & Telha                 & $1,2$                            & $0,06$                      \\
		\hline
		Cobertura & Cimento               & $0,46$                           & $0,04$                      \\
		\hline
		Isolante  & Espuma de Poliuretano & $0,028$                          & $0,17$                      \\
		\hline
		Interior  & Gesso                 & $0,25$                           & $0,03$                      \\
		\hline
	\end{tabular}
\end{center}

% subsection Telhado (end)


\subsection{Portas}\label{sub:Portas}

% subsection Portas (end)


\subsection{Janelas}\label{sub:Janelas}

Por fim, foram idealizadas duas janelas que satisfazem as necessidades térmicas do espaço.
Para tal, optou-se por uma construção de \textbf{duas folhas}, com uma \textbf{estrutura de alumínio} e
vidro duplo.
\begin{center}
	\begin{tabular}{||c c c c||}
		\hline
		Estrutura & Material & $k \hspace{1mm} (Wm^{-1}K^{-1})$ & $\Delta x \hspace{1mm} (m)$ \\ [0.5ex]
		\hline\hline
		Perfil    & Alumínio & $237$                            & $0,053$                     \\
		\hline
		Vidro     & Vidro    & $0,79$                           & $0,004$                     \\
		\hline
		Ar        & Ar       & $0,025$                          & $0,023$                     \\
		\hline
		Vidro     & Vidro    & $0,79$                           & $0,004$                     \\
		\hline
	\end{tabular}
\end{center}

% subsection Janelas (end)


%section Selecao de materiais (end)



\section{Estrutura}\label{sec:Estrutura}


\subsection{Croqui}\label{sub:Croqui}

% subsection Croqui (end)


\subsection{Resistencia Termica nas Seccoes}\label{sub:Resistencia Termica nas Seccoes}

\subsubsection{Zona A}\label{ssub:zonaa}

Tendo em conta os materiais apresentados na secção e, o croqui da estrutura, \textbf{a zona A}, para funcionar
à temperatura de $ 15^\circ C $ possui as seguintes características:

\begin{table}[htpb]
	\begin{center}
		\begin{tabular}{c c c c c}
			\toprule{}
			Secção                     & Material               & $ k \hspace{1mm} (Wm^{-1}K^{-1}$) & $ \Delta x \hspace{1mm} (m)$ & Área $(m^2) $ \\
				\midrule{}

				% Exterior
			\multirow{5}{*}{}          & Cimento                & $0.46$                            & $0.09$                       & $48,5$          \\
				\cline{2-5}
			Parede                     & Poliestireno Expandido & $0.037$                           & $0.02$                       & $48,5$          \\
				\cline{2-5}
			Exterior                   & Betão Armado           & $2$                               & $0.18$                       & $48,5$          \\
				\cline{2-5}
			($\times 2$)               & Poliestireno Expandido & $0.037$                           & $0.02$                       & $48,5$          \\
				\cline{2-5}
			                           & Gesso                  & $0.25$                            & $0.01$                       & $48,5$          \\
				\midrule{}

				% Interior não estrutural
			Parede \multirow{4}{*}{}   & Gesso                  & $0.25$                            & $0.01$                       & $40$          \\
				\cline{2-5}
			Interior                   & Poliestireno Extrudido & $0.033$                           & $0.08$                       & $40$          \\
				\cline{2-5}
			Não Estrutural             & Madeira Pinus          & $0.12$                            & $0.1$                        & $40$          \\
				\cline{2-5}
			($\times 1$)               & Gesso                  & $0.25$                            & $0.01$                       & $40$          \\
				\midrule{}

				% Interior estrutural
			\multirow{5}{*}{}          & Gesso                  & $0.25$                            & $0.01$                       & $37$          \\
				\cline{2-5}
			Parede                     & Poliestireno Extrudido & $0.033$                           & $0.02$                       & $37$          \\
				\cline{2-5}
			Interior                   & Betão Armado           & $2$                               & $0.18$                       & $37$          \\
				\cline{2-5}
			Estrutural ($\times 1$)    & Poliestireno Extrudido & $0.033$                           & $0.02$                       & $37$          \\
				\cline{2-5}
			                           & Gesso                  & $0.25$                            & $0.01$                       & $37$          \\
				\midrule{}

			\multirow{5}{*}{}		   & Alumínio 				& $237$                            	& $0,053$                      & $0,66$		   \\
				\cline{2-5}
			Janela    				   & Vidro    				& $0,79$                           	& $0,004$                      & $1,34$		   \\
				\cline{2-5}
			($\times 1$)			   & Ar       				& $0,025$                          	& $0,023$                      & $1,34$		   \\
				\cline{2-5}
									   & Vidro    				& $0,79$                           	& $0,004$                      & $1,34$		   \\
				\midrule{}
				% Porta
			Porta de Subir ($\times 1$) & Fibra de vidro        & $0.04$                            & $0.1$                        & $15$           \\
			\bottomrule{}
		\end{tabular}
	\end{center}
	\caption{Composição da zona A}\label{tab:zona_a}
\end{table}

Com base na tabela \ref*{tab:zona_a}, o cálculo das resistências para esta sucede-se da seguinte forma:

\paragraph{Janela:}\label{par:zona_a_janela}Os vidros e o ar estão associados em série:

\begin{equation}
	R_{janelas} = \dfrac{1}{\dfrac{1}{R_{aluminio}} + \dfrac{1}{2 \times R_{vidro} + R_{ar}}}
	\label{eq:zona_a_janelas}
\end{equation}

\begin{equation}
	\Leftrightarrow 
	R_{janelas} =
	\dfrac{1}{
		\dfrac{1}{
			\dfrac{0,053}{237 \times 0,66}
		}+
		\dfrac{1}{
			2 \times \dfrac{0.004}{0.79 \times 1.34} +
			\dfrac{0.023}{0.025 \times 1.34}
		}
	}
		= 3.40 \times 10^{-4} \hspace{1mm} KW^{-1}
	\label{eq:zona_a_janelas2}
\end{equation}

% paragraph Janelas (end)

\paragraph{Parede exterior:}\label{par:zona_a_ext}Camadas associadas em série:

\begin{equation}
    R_{parede\_n\tilde{a}o\_ext + porta + janela} =
        \dfrac{1}{
			\dfrac{1}{
			R_{cimento} + 2 \times R_{poliestireno} + R_{bet\tilde{a}o} + R_{gesso}
			}  
			+
			\dfrac{1}{
				R_{porta}
			}  
			+
			\dfrac{1}{
				R_{janela}
			}  
		}
    \label{eq:zona_a_ext_1}
\end{equation}

\begin{equation}
    \Leftrightarrow R =
        \dfrac{1}{
			\dfrac{1}{
				\dfrac{0.09}{0.46 \times 48.5} +
				2 \times \dfrac{0.02}{0.037 \times 48.5} +
				\dfrac{0.18}{2 \times 48.5} +
				\dfrac{0.01}{0.25 \times 48.5} 
        	}
			+
			\dfrac{1}{
            	\dfrac{0.1}{0.04 \times 15}
			}
			+
			\dfrac{1}{
            	3.40 \times 10^{-4}
			}
		}
    \label{eq:zona_a_ext_2}
\end{equation}

\begin{equation}
	\Leftrightarrow R_{parede\_n\tilde{a}o\_ext + porta + janela} = 2.47 \times 10^{-2} \hspace{1mm} KW^{-1}
	\label{eq:zona_a_ext_3}
\end{equation}

\paragraph{Parede Interior N\~ao Estrutural com porta:}\label{par:zona_a_int_n_est}Paralelo entre a parede e a porta. 

\begin{equation}
    R_{parede\_n\tilde{a}o\_estrut + porta} =
        \dfrac{1}{
			\dfrac{1}{
            2 \times R_{gesso} + R_{poliestireno\_extrudido} + R_{madeira\_pinus}
			}  
			+
			\dfrac{1}{
				R_{porta}
			}  
		}
    \label{eq:zona_a_int_n_est_1}
\end{equation}

\begin{equation}
    \Leftrightarrow R_{parede\_n\tilde{a}o\_estrut + porta} =
        \dfrac{1}{
			\dfrac{1}{
            2 \times \dfrac{0.01}{0.25 \times 40} +
            \dfrac{0.08}{0.033 \times 40} +
            \dfrac{0.1}{0.12 \times 40}
        	}
			+
			\dfrac{1}{
            \dfrac{0.1}{0.12 \times 3}
			}
		}
    \label{eq:zona_a_int_n_est_2}
\end{equation}

\begin{equation}
    \Leftrightarrow R_{parede\_n\tilde{a}o\_estrut + porta} =
	\dfrac{1}{
        8.34 \times 10^{-2} + 2.78 \times 10^{-2}
	}
	= 3.61 \times 10^{-2} \hspace{1mm} KW^{-1}
    \label{eq:zona_a_int_n_est_3}
\end{equation}

\paragraph{Parede Interior Estrutural:}\label{par:zona_a_int_est}

\begin{equation}
	R_{parede\_estrut} = 2 \times R_{gesso} + 2 \times R_{poliestireno} + R_{bet\tilde{a}o}
	\label{eq:zona_a_int_est_1}
\end{equation}

\begin{equation}
	\Leftrightarrow R_{parede\_estrut} =
		2 \times \dfrac{0.01}{0.25 \times 37} +
		2 \times \dfrac{0.02}{0.033 \times 37} +
		\dfrac{0.18}{2 \times 37}
		= 3.74 \times 10^{-2} \hspace{1mm} KW^{-1}
	\label{eq:zona_a_int_est_2}
\end{equation}


\paragraph{Total:}\label{par:zona_a_total} Tendo em conta que os componentes est\~ao
associados em série e considerando os resultados das
equa\c{c}\~oes~\ref*{eq:zona_a_ext_3},~\ref*{eq:zona_a_int_n_est_3} e~\ref*{eq:zona_a_int_est_2}

\begin{equation}
	R_{total} =
		2.47 \times 10^{-2} + 3.61 \times 10^{-2} + 6.28 \times 10^{-2} = 1.23 \times 10^{-1} \hspace{1mm} KW^{-1}
	\label{eq:zona_a_total}
\end{equation}


\subsubsection{Zona B}\label{ssub:zonab}

Tendo em conta os materiais apresentados na secção e, o croqui da estrutura, \textbf{a zona B}, para funcionar
à temperatura de $ 20^\circ C $ possui as seguintes características:

\begin{table}[htpb]
	\begin{center}
		\begin{tabular}{c c c c c}
			\toprule{}
			Secção                     & Material               & $ k \hspace{1mm} (Wm^{-1}K^{-1}$) & $ \Delta x \hspace{1mm} (m)$ & Área $(m^2) $ \\
				\midrule{}

				% Exterior
			\multirow{5}{*}{}          & Cimento                & $0.46$                            & $0.09$                       & $57.5$          \\
				\cline{2-5}
			Parede                     & Poliestireno Expandido & $0.037$                           & $0.02$                       & $57.5$          \\
				\cline{2-5}
			Exterior                   & Betão Armado           & $2$                               & $0.18$                       & $57.5$          \\
				\cline{2-5}
			($\times 2$)               & Poliestireno Expandido & $0.037$                           & $0.02$                       & $57.5$          \\
				\cline{2-5}
			                           & Gesso                  & $0.25$                            & $0.01$                       & $57.5$          \\
				\midrule{}

				% Interior não estrutural
			Parede \multirow{4}{*}{}   & Gesso                  & $0.25$                            & $0.01$                       & $25$          \\
				\cline{2-5}
			Interior                   & Poliestireno Extrudido & $0.033$                           & $0.08$                       & $25$          \\
				\cline{2-5}
			Não Estrutural             & Madeira Pinus          & $0.12$                            & $0.1$                        & $25$          \\
				\cline{2-5}
			($\times 1$)               & Gesso                  & $0.25$                            & $0.01$                       & $25$          \\
				\midrule{}

				% Interior estrutural
			\multirow{5}{*}{}          & Gesso                  & $0.25$                            & $0.01$                       & $40$          \\
				\cline{2-5}
			Parede                     & Poliestireno Extrudido & $0.033$                           & $0.02$                       & $40$          \\
				\cline{2-5}
			Interior                   & Betão Armado           & $2$                               & $0.18$                       & $40$          \\
				\cline{2-5}
			Estrutural ($\times 1$)    & Poliestireno Extrudido & $0.033$                           & $0.02$                       & $40$          \\
				\cline{2-5}
			                           & Gesso                  & $0.25$                            & $0.01$                       & $40$          \\
				\midrule{}

			\multirow{5}{*}{}		   & Alumínio 				& $237$                            	& $0,053$                      & $0,66$		   \\
				\cline{2-5}
			Janela    				   & Vidro    				& $0,79$                           	& $0,004$                      & $1,34$		   \\
				\cline{2-5}
			($\times 1$)			   & Ar       				& $0,025$                          	& $0,023$                      & $1,34$		   \\
				\cline{2-5}
									   & Vidro    				& $0,79$                           	& $0,004$                      & $1,34$		   \\
				\midrule{}
				% Porta
			Porta dupla ($\times 1$)   & Madeira Pinus          & $0.12$                            & $0.1$                        & $6$           \\
			\bottomrule{}
		\end{tabular}
	\end{center}
	\caption{Composição da zona B}\label{tab:zona_b}
\end{table}

Com base na tabela \ref*{tab:zona_b}, o cálculo das resistências para esta secção sucede-se da seguinte forma:

\paragraph{Janela:}\label{par:zona_b_janela}Os vidros e o ar estão associados em série:

\begin{equation}
	R_{janelas} = \dfrac{1}{\dfrac{1}{R_{aluminio}} + \dfrac{1}{2 \times R_{vidro} + R_{ar}}}
	\label{eq:zona_b_janelas}
\end{equation}

\begin{equation}
	\Leftrightarrow 
	R_{janelas} =
	\dfrac{1}{
		\dfrac{1}{
			\dfrac{0,053}{237 \times 0,66}
		}+
		\dfrac{1}{
			2 \times \dfrac{0.004}{0.79 \times 1.34} +
			\dfrac{0.023}{0.025 \times 1.34}
		}
	}
		= 3.40 \times 10^{-4} \hspace{1mm} KW^{-1}
	\label{eq:zona_b_janelas2}
\end{equation}

% paragraph Janelas (end)

\paragraph{Parede exterior:}\label{par:zona_b_ext}Camadas associadas em série:

\begin{equation}
    R_{parede\_n\tilde{a}o\_ext + porta + janela} =
        \dfrac{1}{
			\dfrac{1}{
			R_{cimento} + 2 \times R_{poliestireno} + R_{bet\tilde{a}o} + R_{gesso}
			}  
			+
			\dfrac{1}{
				R_{porta}
			}  
			+
			\dfrac{1}{
				R_{janela}
			}  
		}
    \label{eq:zona_b_ext_1}
\end{equation}

\begin{equation}
    \Leftrightarrow R =
        \dfrac{1}{
			\dfrac{1}{
				\dfrac{0.09}{0.46 \times 57.5} +
				2 \times \dfrac{0.02}{0.037 \times 57.5} +
				\dfrac{0.18}{2 \times 57.5} +
				\dfrac{0.01}{0.25 \times 57.5} 
        	}
			+
			\dfrac{1}{
            	\dfrac{0.1}{0.12 \times 6}
			}
			+
			\dfrac{1}{
            	3.40 \times 10^{-4}
			}
		}
    \label{eq:zona_b_ext_2}
\end{equation}

\begin{equation}
	\Leftrightarrow R_{parede\_n\tilde{a}o\_ext + porta + janela} = 3.34 \times 10^{-4} \hspace{1mm} KW^{-1}
	\label{eq:zona_b_ext_3}
\end{equation}

\paragraph{Parede Interior N\~ao Estrutural com porta:}\label{par:zona_b_int_n_est}

\begin{equation}
	R_{parede\_n\tilde{a}o\_estrut} = 2 \times R_{gesso} + R_{poliestireno\_extrudido} + R_{madeira\_pinus}
	\label{eq:zona_b_int_n_est_1}
\end{equation}

\begin{equation}
	\Leftrightarrow R_{parede\_n\tilde{a}o\_estrut} =
	2 \times \dfrac{0.01}{0.25 \times 25} +
	\dfrac{0.08}{0.033 \times 25} +
	\dfrac{0.1}{0.12 \times 25} = 1.34 \times 10^{-1} \hspace{1mm} KW^{-1}
	\label{eq:zona_b_int_n_est_2}
\end{equation}

\paragraph{Parede Interior Estrutural:}\label{par:zona_b_int_est}

\begin{equation}
	R_{parede\_estrut} = 2 \times R_{gesso} + 2 \times R_{poliestireno} + R_{bet\tilde{a}o}
	\label{eq:zona_b_int_est_1}
\end{equation}

\begin{equation}
	\Leftrightarrow R_{parede\_estrut} =
		2 \times \dfrac{0.01}{0.25 \times 40} +
		2 \times \dfrac{0.02}{0.033 \times 40} +
		\dfrac{0.18}{2 \times 40}
		= 3.46 \times 10^{-2} \hspace{1mm} KW^{-1}
	\label{eq:zona_b_int_est_2}
\end{equation}


\paragraph{Total:}\label{par:zona_b_total} Tendo em conta que os componentes est\~ao
associados em série e considerando os resultados das
equa\c{c}\~oes~\ref*{eq:zona_b_ext_3},~\ref*{eq:zona_b_int_n_est_2} e~\ref*{eq:zona_b_int_est_2}

\begin{equation}
	R_{total} =
		3.34 \times 10^{-4} + 1.34 \times 10^{-1} + 3.46 \times 10^{-2} = 1.69 \times 10^{-1} \hspace{1mm} KW^{-1}
	\label{eq:zona_b_total}
\end{equation}


\subsubsection{Zona C}\label{ssub:Zona C}

Tendo em conta os materiais apresentados na secção e, o croqui da estrutura, \textbf{a zona C}, para funcionar
à temperatura de $ -10^\circ C $ possui as seguintes características:

\begin{table}[htpb]
	\begin{center}
		\begin{tabular}{c c c c c}
			\toprule{}
			Secção                     & Material               & $ k \hspace{1mm} (Wm^{-1}K^{-1}$) & $ \Delta x \hspace{1mm} (m)$ & Área $(m^2) $ \\
				\midrule{}

				% Exterior
			\multirow{5}{*}{}          & Cimento                & $0.46$                            & $0.09$                       & $40$          \\
				\cline{2-5}
			Parede                     & Poliestireno Expandido & $0.037$                           & $0.02$                       & $40$          \\
				\cline{2-5}
			Exterior                   & Betão Armado           & $2$                               & $0.18$                       & $40$          \\
				\cline{2-5}
			($\times 1$)               & Poliestireno Expandido & $0.037$                           & $0.02$                       & $40$          \\
				\cline{2-5}
			                           & Gesso                  & $0.25$                            & $0.01$                       & $40$          \\
				\midrule{}

				% Interior não estrutural
			Parede \multirow{4}{*}{}   & Gesso                  & $0.25$                            & $0.01$                       & $50$          \\
				\cline{2-5}
			Interior                   & Poliestireno Extrudido & $0.033$                           & $0.08$                       & $50$          \\
				\cline{2-5}
			Não Estrutural             & Madeira Pinus          & $0.12$                            & $0.1$                        & $50$          \\
				\cline{2-5}
			($\times2$)                & Gesso                  & $0.25$                            & $0.01$                       & $50$          \\
				\midrule{}

				% Interior estrutural
			\multirow{5}{*}{}          & Gesso                  & $0.25$                            & $0.01$                       & $37$          \\
				\cline{2-5}
			Parede                     & Poliestireno Extrudido & $0.033$                           & $0.02$                       & $37$          \\
				\cline{2-5}
			Interior                   & Betão Armado           & $2$                               & $0.18$                       & $37$          \\
				\cline{2-5}
			Estrutural ($\times 1$)    & Poliestireno Extrudido & $0.033$                           & $0.02$                       & $37$          \\
				\cline{2-5}
			                           & Gesso                  & $0.25$                            & $0.01$                       & $37$          \\
				\midrule{}

				% Porta
			Porta Simples ($\times 1$) & Madeira Pinus          & $0.12$                            & $0.1$                        & $3$           \\
			\bottomrule{}
		\end{tabular}
	\end{center}
	\caption{Composição da zona C}\label{tab:zona_c}
\end{table}


Com base na tabela \ref*{tab:zona_c}, o cálculo das resistências para esta secção sucede-se da seguinte forma:

\paragraph{Parede exterior:}\label{par:zona_c_ext}Camadas associadas em série:

\begin{equation}
	R_{parede\_ext} = R_{cimento} + 2 \times R_{poliestireno} + R_{bet\tilde{a}o} + R_{gesso}
	\label{eq:zona_c_ext_1}
\end{equation}

\begin{equation}
	\Leftrightarrow R_{parede\_ext} =
	\dfrac{0.09}{0.46 \times 40} +
	2 \times \dfrac{0.02}{0.037 \times 40} +
	\dfrac{0.18}{2 \times 40} +
	\dfrac{0.01}{0.25 \times 40} = 3.52 \times 10^{-2} \hspace{1mm} KW^{-1}
	\label{eq:zona_c_ext_2}
\end{equation}

% paragraph Parede Exterior (end)

\paragraph{Paredes Interiores Não Estruturais:}\label{par:zona_c_int_n_est}

\begin{equation}
	R_{parede\_n\tilde{a}o\_estrut} = 2 \times R_{gesso} + R_{poliestireno\_extrudido} + R_{madeira\_pinus}
	\label{eq:zona_c_int_n_est_1}
\end{equation}

\begin{equation}
	\Leftrightarrow R_{parede\_n\tilde{a}o\_estrut} =
	2 \times \dfrac{0.01}{0.25 \times 50} +
	\dfrac{0.08}{0.033 \times 50} +
	\dfrac{0.1}{0.12 \times 50} = 6.67 \times 10^{-2} \hspace{1mm} KW^{-1}
	\label{eq:zona_c_int_n_est_2}
\end{equation}

\paragraph{Parede Interior Estrutural com porta:}\label{par:zona_c_int_est}Paralelo entre a parede e a porta.

\begin{equation}
	R_{parede\_estrut + porta} =
	\dfrac{1}{
		2 \times R_{gesso} + 2 \times R_{poliestireno} + R_{bet\tilde{a}o}
	}
	+
	\dfrac{1}{
		R_{porta}
	}
	\label{eq:zona_c_int_est_1}
\end{equation}

\begin{equation}
	\Leftrightarrow R_{parede\_estrut + porta} =
	\dfrac{1}{
		2 \times \dfrac{0.01}{0.25 \times 37} +
		2 \times \dfrac{0.02}{0.033 \times 37} +
		\dfrac{0.18}{2 \times 37}
	}
	+
	\dfrac{1}{
		\dfrac{0.1}{0.12 \times 3}
	}
	\label{eq:zona_c_int_est_2}
\end{equation}

\begin{equation}
	\Leftrightarrow R_{parede\_estrut + porta} = 3.29 \times 10^{-2} \hspace{1mm} KW^{-1}
	\label{eq:zona_c_int_est_3}
\end{equation}

\paragraph{Total:}\label{par:zona_c_total:} Tendo em conta que os componentes est\~ao
associados em série e considerando os resultados das
equa\c{c}\~oes~\ref{eq:zona_c_ext_2},~\ref{eq:zona_c_int_n_est_2} e~\ref{eq:zona_c_int_est_3}

\begin{equation}
	R_{total} =
			3.52 \times 10^{-2} + 6.67 \times 10^{-2} + 3.29 \times 10^{-2} = 1.34 \times 10^{-1} \hspace{1mm} KW^{-1}
	\label{eq:zona_c_total}
\end{equation}
% subsubsection Zona C (end)


\subsubsection{Zona D}\label{ssub:Zona D}

% subsubsection Zona D (end)


\subsubsection{Zona E}\label{ssub:Zona E}

Tendo em conta os materiais apresentados na secção e, o croqui da estrutura, \textbf{a zona E}, para funcionar
à temperatura de $ 10^\circ C $ possui as seguintes características:

\begin{table}[htpb]
	\begin{center}
		\begin{tabular}{c c c c c}
			\toprule{}
			Secção                     & Material               & $ k \hspace{1mm} (Wm^{-1}K^{-1}$) & $ \Delta x \hspace{1mm} (m)$ & Área $(m^2) $ \\
				\midrule{}

				% Exterior
			\multirow{5}{*}{}          & Cimento                & $0.46$                            & $0.09$                       & $90$          \\
				\cline{2-5}
			Paredes                    & Poliestireno Expandido & $0.037$                           & $0.02$                       & $90$          \\
				\cline{2-5}
			Exteriores                 & Betão Armado           & $2$                               & $0.18$                       & $90$          \\
				\cline{2-5}
			(Área com base no croqui)  & Poliestireno Expandido & $0.037$                           & $0.02$                       & $90$          \\
				\cline{2-5}
			                           & Gesso                  & $0.25$                            & $0.01$                       & $90$          \\
				\midrule{}

				% Interior não estrutural
			Parede \multirow{4}{*}{}   & Gesso                  & $0.25$                            & $0.01$                       & $47$          \\
				\cline{2-5}
			Interior                   & Poliestireno Extrudido & $0.033$                           & $0.08$                       & $47$          \\
				\cline{2-5}
			Não Estrutural             & Madeira Pinus          & $0.12$                            & $0.1$                        & $47$          \\
				\cline{2-5}
			($\times1$)                & Gesso                  & $0.25$                            & $0.01$                       & $47$          \\
				\midrule{}

				% Porta
			Porta Simples ($\times 1$) & Madeira Pinus          & $0.12$                            & $0.1$                        & $3$           \\
			\bottomrule{}
		\end{tabular}
	\end{center}
	\caption{Composição da zona E}\label{tab:zona_e}
\end{table}

Com base na tabela \ref*{tab:zona_e}, o cálculo das resistências para esta secção sucede-se da seguinte forma:

\paragraph{Parede exterior:}\label{par:zona_e_ext}Camadas associadas em série:

\begin{equation}
	R_{parede\_ext} = R_{cimento} + 2 \times R_{poliestireno} + R_{bet\tilde{a}o} + R_{gesso}
	\label{eq:zona_e_ext_1}
\end{equation}

\begin{equation}
	\Leftrightarrow R_{parede\_ext} =
	\dfrac{0.09}{0.46 \times 90} +
	2 \times \dfrac{0.02}{0.037 \times 90} +
	\dfrac{0.18}{2 \times 90} +
	\dfrac{0.01}{0.25 \times 90} = 1.56 \times 10^{-2} \hspace{1mm} KW^{-1}
	\label{eq:zona_e_ext_2}
\end{equation}

\paragraph{Parede Interior N\~ao Estrutural com porta:}\label{par:zona_e_int_n_est}Paralelo entre a parede e a porta. 

\begin{equation}
    R_{parede\_n\tilde{a}o\_estrut + porta} =
        \dfrac{1}{
			\dfrac{1}{
            2 \times R_{gesso} + R_{poliestireno\_extrudido} + R_{madeira\_pinus}
			}  
			+
			\dfrac{1}{
				R_{porta}
			}  
		}
    \label{eq:zona_e_int_n_est_1}
\end{equation}

\begin{equation}
    \Leftrightarrow R_{parede\_n\tilde{a}o\_estrut + porta} =
        \dfrac{1}{
			\dfrac{1}{
            2 \times \dfrac{0.01}{0.25 \times 47} +
            \dfrac{0.08}{0.033 \times 47} +
            \dfrac{0.1}{0.12 \times 47}
        	}
			+
			\dfrac{1}{
            \dfrac{0.1}{0.12 \times 3}
			}
		}
    \label{eq:zona_e_int_n_est_2}
\end{equation}

\begin{equation}
    \Leftrightarrow R_{parede\_n\tilde{a}o\_estrut + porta} =
	\dfrac{1}{
        7.10 \times 10^{-2} + 2,78 \times 10^{-2}
	}
	= 5.66 \times 10^{-2} \hspace{1mm} KW^{-1}
    \label{eq:zona_e_int_n_est_3}
\end{equation}


\paragraph{Total:}\label{par:zona_e_total:} Tendo em conta que os componentes est\~ao
associados em série e considerando os resultados das
equa\c{c}\~oes~\ref{eq:zona_e_ext_2} e ~\ref{eq:zona_e_int_n_est_3}

\begin{equation}
	R_{total} =
			1.56 \times 10^{-2} + 5.65 \times 10^{-2} = 7.21 \times 10^{-2} \hspace{1mm} KW^{-1}
	\label{eq:zona_e_total}
\end{equation}
% paragraph Parede Interior Não Estrutural com porta (end)

% subsubsection Zona E (end)

% subsection Resistencia Termica nas Seccoes (end)

\subsubsection{Telhado}\label{ssub:Telhado}

Para cada uma das secções, \textbf{a área do telhado é igual}. Deste modo, ao calcular a resistência do telhado 
para uma secção, estamos a obter também o seu valor para todas as outras.
\begin{table}[htpb]
	\begin{center}
		\begin{tabular}{c c c c c}
			\toprule{}
			Secção                     & Material               & $ k \hspace{1mm} (Wm^{-1}K^{-1}$) & $ \Delta x \hspace{1mm} (m)$ & Área $(m^2) $ \\
				\midrule{}

			\multirow{5}{*}{}          & Telha                  & $1.2$                             & $0.06$                       & $80.1$        \\
				\cline{2-5}
			Telhado                    & Cimento				& $0.46$                            & $0.04$                       & $80.1$        \\
				\cline{2-5}
			                           & Espuma de Poliuretano  & $0.028$                           & $0.17$                       & $80.1$        \\
				\cline{2-5}
									   & Gesso				 	& $0.25$                            & $0.03$                       & $80.1$        \\
			\bottomrule{}
		\end{tabular}
	\end{center}
	\caption{Composição da zona C}\label{tab:telhado}
\end{table}

Com base na tabela \ref*{tab:telhado}, o cálculo das resistência para o telhado sucede-se da seguinte forma:

\begin{equation}
	R_{telhado} = R_{telha} + R_{cimento} + R_{poliuretano} + R_{gesso}
	\label{eq:telhado1}
\end{equation}

\begin{equation}
	\Leftrightarrow R_{telhado} =
	\dfrac{0.06}{1.2 \times 80.1} +
	\dfrac{0.04}{0.46 \times 80.1} +
	\dfrac{0.17}{0.028 \times 80.1} +
	\dfrac{0.03}{0.25 \times 80.1} = 7.90 \times 10^{-2} \hspace{1mm} KW^{-1}
	\label{eq:telhado2}
\end{equation}


% section Estrutura (end)


\printbibliography{}

\end{document}
